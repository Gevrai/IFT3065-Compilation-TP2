%%%%%%%%%%%%%%%%%%%%%%%%%%%%%%%%%%%%%%%%%
% Beamer Presentation
% LaTeX Template
% Version 1.0 (10/11/12)
%
% This template has been downloaded from:
% http://www.LaTeXTemplates.com
%
% License:
% CC BY-NC-SA 3.0 (http://creativecommons.org/licenses/by-nc-sa/3.0/)
%
%%%%%%%%%%%%%%%%%%%%%%%%%%%%%%%%%%%%%%%%%

%----------------------------------------------------------------------------------------
%	PACKAGES AND THEMES
%----------------------------------------------------------------------------------------

\documentclass{beamer}

\usepackage[utf8]{inputenc}
\usepackage{multicol}

\mode<presentation> {

% The Beamer class comes with a number of default slide themes
% which change the colors and layouts of slides. Below this is a list
% of all the themes, uncomment each in turn to see what they look like.

%\usetheme{default}
%\usetheme{AnnArbor}
%\usetheme{Antibes}
%\usetheme{Bergen}
%\usetheme{Berkeley}
%\usetheme{Berlin}
%\usetheme{Boadilla}
%\usetheme{CambridgeUS}
%\usetheme{Copenhagen}
%\usetheme{Darmstadt}
%\usetheme{Dresden}
%\usetheme{Frankfurt}
%\usetheme{Goettingen}
%\usetheme{Hannover}
%\usetheme{Ilmenau}
%\usetheme{JuanLesPins}
%\usetheme{Luebeck}
% \usetheme{Madrid}
\usetheme{Malmoe}
%\usetheme{Marburg}
%\usetheme{Montpellier}
%\usetheme{PaloAlto}
%\usetheme{Pittsburgh}
%\usetheme{Rochester}
%\usetheme{Singapore}
%\usetheme{Szeged}
%\usetheme{Warsaw}

% As well as themes, the Beamer class has a number of color themes
% for any slide theme. Uncomment each of these in turn to see how it
% changes the colors of your current slide theme.

% \usecolortheme{albatross}
% \usecolortheme{beaver}
%\usecolortheme{beetle}
%\usecolortheme{crane}
%\usecolortheme{dolphin}
\usecolortheme{dove}
%\usecolortheme{fly}
%\usecolortheme{lily}
%\usecolortheme{orchid}
%\usecolortheme{rose}
%\usecolortheme{seagull}
%\usecolortheme{seahorse}
%\usecolortheme{whale}
%\usecolortheme{wolverine}

%\setbeamertemplate{footline} % To remove the footer line in all slides uncomment this line
%\setbeamertemplate{footline}[page number] % To replace the footer line in all slides with a simple slide count uncomment this line

\setbeamertemplate{navigation symbols}{} % To remove the navigation symbols from the bottom of all slides uncomment this line
}

\usepackage{graphicx} % Allows including images
\usepackage{booktabs} % Allows the use of \toprule, \midrule and \bottomrule in tables

%----------------------------------------------------------------------------------------
%	TITLE PAGE
%----------------------------------------------------------------------------------------

\title[Optimisations Typer]{Optimisations de Typer\\ Constant Folding et Propagation
  de constantes} % The short title appears at the bottom of every slide, the full title is only on the title page

\author{Gevrai Jodoin-Tremblay \& Nicolas Lafond} % Your name
\institute[UDM] % Your institution as it will appear on the bottom of every slide, may be shorthand to save space
{
Université de Montréal \\ % Your institution for the title page
\medskip
}
\date{Lundi 27 Mars 2017} % Date, can be changed to a custom date

\begin{document}

\begin{frame}
\titlepage % Print the title page as the first slide
\end{frame}

% \begin{frame }
% \frametitle{Overview} % Table of contents slide, comment this block out to remove it
% \tableofcontents % Throughout your presentation, if you choose to use \section{} and \subsection{} commands, these will automatically be printed on this slide as an overview of your presentation
% \end{frame}

%----------------------------------------------------------------------------------------
%	PRESENTATION SLIDES
%----------------------------------------------------------------------------------------

%------------------------------------------------
\section{Optimisations} % Sections can be created in order to organize your presentation into discrete blocks, all sections and subsections are automatically printed in the table of contents as an overview of the talk
%------------------------------------------------
\subsection{Choix d'optimisations} 

\begin{frame}{Choix d'optimisations}

\pause

Initialement: 
  \begin{itemize}
    \item Inlining de fonction: Nécessite une certaine forme d'analyse du programme
    \item Elimination de code mort: Problématique avec les effets de bords
  \end{itemize}
 
\pause

Choix finaux:
  \begin{itemize}
    \item Constant folding: De base, ouvre plusieurs possibilités pour d'autres optimisations
    \item Propagation de constantes: Language purement fonctionnel, on ne
      travaille d'une certaine façon que sur des constantes (immuables)
  \end{itemize}

  Constant folding et propagation de constantes se complémentent extrêmement bien
\end{frame}

%---------------------------------------------------------------
\subsection{Constant Folding} 
%---------------------------------------------------------------

\begin{frame}{Cas de base - Appels sur Entiers et Flottants}
Opérations triviales 
  \begin{multicols}{3}
  \begin{itemize}
    \item $e + 0 \Rightarrow e $
    \item $e - 0 \Rightarrow e $
    \item $e * 1 \Rightarrow e $
    \item $e \div 1 \Rightarrow e $
    \item $e * 0 \Rightarrow 0 $
    \item $0 \div e \Rightarrow 0 $
  \end{itemize}
  \end{multicols}

Précalcul d'expressions
  \begin{multicols}{2}
  \begin{itemize}
    \item $ 1 + 2 \Rightarrow 3 $
    \item $ 2 - 2 \Rightarrow 0 $
    \item $ 2 * 2 \Rightarrow 4 $
    \item $ 2 / 2 \Rightarrow 1 $
    \item $ Int\_>$ 1 2 $\Rightarrow$ \emph{false}
    \item $ Int\_<$ 1 2 $\Rightarrow$ \emph{true}
    \item $ Int\_eq$ 1 2 $\Rightarrow$ \emph{false}
    \item $ Int\_>=$ 1 2 $\Rightarrow$ \emph{false}
    \item $ Int\_<=$ 1 2 $\Rightarrow$ \emph{true}
  \end{itemize}
  \end{multicols}

\end{frame}

%---------------------------------------------------------------
\begin{frame}{Cas de base - Autres elexp}
Autres opérations
  \begin{itemize}
    \item Float\_to\_string 1 $\Rightarrow$ '1'
    \item String\_eq 'a' 'a' $\Rightarrow$ true 
    \item case true $\mid$ true $\rightarrow$ e $\mid$ \_ $\rightarrow$ ... \\ $\Rightarrow$ e
  \end{itemize}
\end{frame}
%---------------------------------------------------------------
\begin{frame}{Version Naïve}
  Propager l'optimisation dans les enfants
  \begin{itemize}
  \item Elexp.Case(l, e, branches, default)
    \\ $\Rightarrow$ Elexp.Case(l, cstfld e, cstfld branches, cstfld default)
  \item Elexp.Lambda(name, e) \\ $\Rightarrow$ Elexp.Lambda(name, cstfld e)
  \item Elexp.Let(l, nexprs, body) \\ $\Rightarrow$ Elexp.Let(l, cstfld nexprs, cstfld body)
  \end{itemize}
\pause
N'affecte que les feuilles!
\end{frame}
%---------------------------------------------------------------
\begin{frame}{Version Simple et complète}
  La fonction \emph{cstfld} renvoie un boolean \emph{true} si changée \\
  $\rightarrow$ \emph{optimize} rappelle \emph{cstfld} dans ce cas.
  \pause
  \begin{itemize}

    \item Avantages
      \begin{itemize}
        \item Facile à implémenter
        \item Renvoie un elexp complètement \emph{fold}
        \item \emph{optimize} peut rappeler la propagation de constante
      \end{itemize}
  \pause
    \item Désavantages
      \begin{itemize}
        \item Complexité algorithmique en pire cas de O($n^2$)
        \item Rappel des autres optimisations après chaque passe
      \end{itemize}

  \end{itemize}
\end{frame}

\begin{frame}{Notre solution - One pass optimization}
  Chaque noeud rappelle l'optimisation si changement. \\
  $\rightarrow$ Besoin d'une seule passe à travers l'arbre de elexp
  % \\ Boolean 'global' renvoyé: changement à eu lieu n'importe où?
  \pause
  \begin{itemize}

    \item Avantages
      \begin{itemize}
        \item Complexité algorithmique en pire cas de O($2n$)
        \item \emph{optimize} peut rappeler la propagation de constante plus
          efficacement (\emph{fold} complet)
      \end{itemize}
  \pause
    \item Désavantages
      \begin{itemize}
        \item Changement global représenté par une variable mutable
        \item Cas plus difficiles à gérer
      \end{itemize}

  \end{itemize}
\end{frame}

%---------------------------------------------------------------
\subsection{Propagation de constantes} 
%---------------------------------------------------------------

\begin{frame}
\frametitle{Propagation de constantes}
Comment faire la propagation de constantes?\\
Par exemple sur le code suivant:\\
$x = 1;$\\
\bigskip
$y = x + 2;$\\
\bigskip
on veut que la deuxième expression devienne \\
$y = 1 + 2;$\\

\end{frame}
%---------------------------------------------------------------
\begin{frame}
\frametitle{Solution}
    \begin{itemize}
        \item Lorsqu'on a une expression de type variable on cherche dans le contexte si cette variable a été définie.
        \item Si oui alors on vérifie la valeur associé à cette variable et on retourne une expression correspondante à cette valeur.
        \item sinon on retourne l'expression tel quelle.
    \end{itemize}
\end{frame}
%---------------------------------------------------------------
\begin{frame}
    \frametitle{Quelques cas problématiques}
    x = 1; \\
    \bigskip
    f : Int -$>$ Int;\\
    f x = x * x;\\
    \bigskip
    g = let x = 3 in x * 2;\\
    \bigskip
    Solution : il faut enlever du contexte la paire de x avec la référence vers sa valeur lorsqu'on fait un appel récursive
    de la fonction sur le body de la fonction ou du let.
\end{frame}
%---------------------------------------------------------------
\section{Résultats}
\subsection{Tests de performance}
%---------------------------------------------------------------
\begin{frame}
    \frametitle{Deuxième script de tests}
    foo : Int -$>$ Int;\\
    foo x = case true
        $|$ true =$>$ x + 0 - 4 + 3 ...\\
    \bigskip
    \begin{itemize}
      \item Temp moyen sur 10 tests avec foo appelé 10000 fois
      \item Chaque cas de base sur entiers est présent
      \item Test le folding de constantes, ainsi que le précalcul du \emph{'case'}
    \end{itemize}

    \begin{table}
      \begin{tabular}{l l l}
        \toprule
        \textbf{Optimisation} & \textbf{Temp moyen (s)} & \textbf{Gain de performance}\\
        \midrule
        Aucune & 0.64 & - \\
        Constant Folding & 0.43 & 33\% \\
        Constant Propagation & 0.64 & 0\% \\
        Les deux & 0.43 & 33\% \\
        \bottomrule
      \end{tabular}
      \caption{Résultat des tests sur le script 2}
    \end{table}
\end{frame}
%---------------------------------------------------------------
\begin{frame}
    \frametitle{Premier script de tests}
    const1 = 100; const2 = 200; const3 = 200; ...\\
    foo x = x + const1 * const2 - const3 / ...\\
    \bigskip
    \begin{itemize}
      \item Temp moyen sur 10 tests avec foo appelé 10000 fois
      \item 24 constantes définies et utilisés pour le calcul de foo
      \item Test principalement la propagation de constantes, mais aussi la
        combinaison des optimisations
    \end{itemize}

    \begin{table}
      \begin{tabular}{l l l}
        \toprule
        \textbf{Optimisation} & \textbf{Temp moyen (s)} & \textbf{Gain de performance}\\
        \midrule
        Aucune & 0.54 & - \\
        Constant Folding & 0.54 & 0\% \\
        Constant Propagation & 0.444 & 18\% \\
        Les deux & 0.26 & 52\% \\
        \bottomrule
      \end{tabular}
      \caption{Résultat des tests sur le script 1}
    \end{table}
\end{frame}
%---------------------------------------------------------------
\section{Conclusion}
%---------------------------------------------------------------
\begin{frame}
\frametitle{Plans futurs}

  \begin{itemize}
    \item Élimination de code mort
      \begin{itemize}
        \item 'Case' déjà entamé
        \item Potentiel accru après les optimisations déjà implémentées
      \end{itemize}
    \bigskip
    \pause
    \item Inlining de fonctions
      \begin{itemize}
        \item Extension de la propagation de constantes
        \item Spécialisation en place grâce aux autres optimisations
        \item Attention au peeling (récursion) et explosion de code
      \end{itemize}
  \end{itemize}
\end{frame}

\end{document} 

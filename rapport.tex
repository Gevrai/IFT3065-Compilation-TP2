\documentclass{article}

\usepackage[utf8]{inputenc}
\usepackage{url}

\begin{document}

\title{Travail pratique \#2 - IFT-3065}
\author{Nicolas Lafond et Gevrai Jodoin-Tremblay}
\date{Mercredi 22 Mars 2017}
\maketitle

\section{Introduction}

Définitivement, au moins 75\% du travail dans ce TP fut de comprendre les
'elexp' que l'on recevait en argument de la fonction 'optimize', et par
le fait même, la structure interne de l'interpréteur \emph{Typer}.

\paragraph{}
Nous avons choisi des optimisations qui se complémentaient le plus possible,
c'est à dire le 'constant folding' ainsi que la propagation de constantes.
Ces deux types d'optimisations sont très complémentaires, puisqu'après chaque
passe de l'une des optimisations, il y a de bonnes chances que ceci ait ouvert
de nouvelles opportunitées pour l'autre.

\paragraph{}
Ainsi, l'un des problèmes auquel nous avons fait face était l'ordonnancement
de ces optimisations, mais aussi de s'assurer qu'il ne reste plus rien à
'fold' ou propager.

\paragraph{}
Ensuite, nous avons tenté l'élimination de code mort dans les 'case', sur des
'cons' simple. En fait, nous avons réellement vu cette élimination comme
une sorte de 'folding', puisque l'on précalcule la branche à l'avance si le
case se fait sur une constante. Cette optimisation n'est en général utile
qu'après la propagation et le 'folding' des constantes et c'est pourquoi nous 
l'avons choisi.

\section{Constantes et variables immuables}
\subsection{Constant folding}

Dans sa forme la plus simple, cette optimisation est très facile à implémenter.
En effet, nous avons des cas de base, transformant une expression 'Elexp.Call'
contenant potentiellement des constantes 'Elexp.Imm', en une simple constante
immuable (précomputation) ou bien en une expression de base (élimination d'une
addition par zéro, d'une mutiplication par un,etc) ou même en une constante
égale à zéro (multiplication par zéro). Ceci se résume en une
vingtaine de cas qui sont véritablement le coeur du 'folding' de constante.

\paragraph{}
Par contre, il faut évidemment généraliser ceci dans les autres types de
'elexp', qui peuvent aussi contenir d'autre 'elexp' imbriqués, nécessitant
d'appeler récursivement l'optimisation sur ceux-ci. C'est ici que les problèmes
peuvent survenir, car une passe de cet algorithme ne fera évidemment que 'fold'
les feuilles de l'arbre de 'elexps'. Il faut ainsi passer plusieurs fois à
travers l'arbre pour bien s'assurer que tous les 'fold' ont bel et bien eu
lieu. Notre solution initiale était donc de renvoyer une valeure boolean 'true'
si un changement à eu lieu n'importe où dans la passe courrante. Si oui, on
recommence jusqu'à ce qu'une passe renvoie 'false' et ainsi on a terminé.
Évidement, ceci peut s'avérer très inneficace, étant O($n^2$) en pire cas.

\paragraph{}
Dans le but d'améliorer cette complexité, dès qu'un appel à \emph{constantfold}
se rend compte qu'il y a eut un changement dans l'un de ces enfants, il retente
une \emph{shallow optimization}. En effet, une nouvelle opportunitée
d'optimisation ne peut être possible que dans un enfant immédiat, puisque chacun
s'assure de bel et bien d'en effectuer le maximum. Ainsi, la complexité tombe à
O($n$) en pire cas, car il ne suffit qu'une seule passe et visite à chacun des
noeuds.

\paragraph{}
Pour ce qui est de gérer un cas comme celui de type : \textbf{if} true
\textbf{then} e1 \textbf{else} e2 $=>$ e1, le langage Typer n'a pas de
syntaxe pour les if toutefois il est possible d'arriver au même effet
avec les Case. Nous avons donc implémenté ce type d'optimisation pour les Case,
en généralisant par le fait même cette optimisation sur les 'cons' simple,
c'est à dire seul.

\subsection{Propagation de constante}

La propagation de constante était plux complexe à réaliser que le "constant
folding" étant donné qu'il fallait également comprendre comment utiliser le 
contexte pour retrouver les valeurs affecté aux variables et transformer les
expressions de manière adéquate en conséquence. Le contexte est une liste dont
les éléments sont des tuples composés d'une chaîne de caractère optionnelle(
pour le nom de la variable ou fonction) et une référence vers un value\_type
dont la définition se trouve dans le fichier "env.ml". Les value\_type sont
les valeurs des expressions obtenues après l'évaluation. La liste du contexte
en elle-même est implémenté dans le fichier "myers.ml" ainsi que les fonctions
nécessaire pour l'utiliser. Pour trouver un élément de la liste avec une
string donné nous avons dû écrire des fonctions qui parcourt la liste un 
élément à la fois à partir du début de la liste car la fonction 'find' défini
dans le fichier "myers.ml" effectue une recherche dichotomique sur la liste ce
qui n'est pas approprié pour le contexte étant donné que les élément ne sont
pas triés. De plus la propagation de constante doit s'effectuer de manière 
récursive sur ses sous-expressions comme c'est le cas pour le constant folding.
Les types d'expression qui peuvent être propagé sont les entiers, les nombres
à virgule flottants, les string et les booléens.

\paragraph{}
Il est également intéressant de constater que la propagation de constante est
très similaire au inlining de fonctions. Nous n'avons pas implémenté cette
optimisation étant donné que nous n'en avions que deux à faire toutefois si
nous devions la faire il aurait fallu agrandir la fonction principale de
la propagation de constante et ajouter comme cas que lorsqu'on reçoit un elexp
de type Call on fait un lookup dans le contexte et si on le trouve avec comme
valeur un type EN.Closure alors on retournerait le body de cette fermeture dans
lequel on aurait propagé son paramètre avec l'expression du deuxième argument
du Call. Évidemment, nous aurions dû faire une certaine analyse du code pour
s'assurer de ne pas faire cette optimisation sur une fonction récursive, se qui
causerait une explosion du code, ou bien sur une fonction trop grande, rendant
le coût de l'appel négligeable.

\section{Évaluation de performances}

Pour tester la diminition du temps d'exécution engendré par les optimisations
nous avons écrit un script en langage typer qui définit une vingtaine de
constantes, une fonction qui effectue un gros calcul arithmétique avec l'aide
de ces constantes et puis finalement une autre fonction qui va effectuer de
manière récursive des appels à la première fonction. Nous avons ensuite 
mesuré le temps d'exécution d'un appel à cette deuxième fonction avec comme
paramètre un très grand nombre et avons fait la comparaison avec et sans les
optimisations. En moyenne, nous avons obtenue que le temps d'exécution est
deux fois plus
grand sans les optimisations que avec celles-ci soit 0.26 secondes avec les 
optimisations et 0.54 sans les optimisations sur une machine avec une
processeur Intel i5-5200U. Si on ne fait que l'optimisation de la propagation
de constante alors on obtient un temps d'exécution de 0.444 secondes et si
on ne fait que le constant folding le temps d'exécution est de 0.54 secondes 
soit le même temps que sans les optimisation ce qui est normal étant donné que
il n'y a aucune possibilité de folding si les constantes ne sont pas propagés.
On voit donc que dans ce scripts le constant folding en lui-même est peu utile
mais que combiné avec la propagation il permet d'obtenir un meilleur résultat
que la propagation de constante seule. Le fichier de script pour ces tests
se nomme "testloop.typer"

\paragraph{}
Nous avons également fait un autre script Typer afin de tester l'amélioration
de performance apporté par le constant folding sans propagation de constante.
Ce script profite aussi de l'optimisation du constant folding sur le Case.
Toujours sur le même processeur nous avons obtenu des temps de
l'ordre de 0.43 avec les optimisations et sans les optimisations les temps
d'exécution sont de l'ordre de 0.64 secondes. Le fichier pour ces test 
s'appelle "folding\_test.typer".


\begin{thebibliography}{9}

\bibitem{realworldocaml}
	Yaron Minsky, Anil Madhavapeddy and Jason Hickey,
	Real World OCaml,
	\url{http://realworldocaml.org},
	O'Reilly,
	2012.

\bibitem{ocaml}
	Learn OCaml
	\url{https://ocaml.org/learn/},

\bibitem{inria}
	Xavier Leroy et al.,
	OCaml Doc and user's manual,
	\url{http://caml.inria.fr/pub/docs/manual-ocaml},
	Institut National de Recherche en Informatique et en Automatique,
	2013.

\end{thebibliography}
\end{document}
